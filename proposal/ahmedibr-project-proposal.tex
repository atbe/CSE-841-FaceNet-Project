\documentclass{article}
\usepackage{hyperref}

\begin{document}

\date{17.~October 2018}

\title{CSE 841 Project Proposal - Naive FaceNet Implementation}
\author{Ibrahim Ahmed}

\maketitle

\section{Scientific Question}

How accurate can a machine learning model trained using a deep neural network predict faces?

\section{Project Description}

FaceNet is a unified embedding for face recognition and clustering developed by employees at Google in 2015. The system proposed, called FaceNet, learns a mapping from face images to a compact Euclidean space where distances directly correspond to a measure of face similarity. The methods used in FaceNet involve a deep convolutional neural network that is trained to directly optimize the embeddings.

Implementing FaceNet will introduce quite a few challenges where I will learn how to use deep neural networks and the TensorFlow framework as well as implementing novel ideas from a white-paper.

You can find the FaceNet paper \href{https://arxiv.org/abs/1503.03832}{here}.

I plan to provide an implementation with visualizations of my results based on its performance on the Labeled Faces in the Wild (LFW) dataset.

\section{Implementation Details}

The naive FaceNet implementation will be written in Python using the \href{https://www.tensorflow.org/}{TensorFlow} machine learning framework.

The implementation will be written on a \href{https://colab.research.google.com}{Google Colaboratory Notebook} which provides free GPU access and a Python runtime, free of charge.

\end{document}